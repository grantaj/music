\documentclass{article}
\usepackage{amsmath}
\usepackage{amsthm}

\newtheorem{lemma}{Lemma}
\newtheorem{theorem}{Theorem}
\newtheorem{definition}{Definition}

\title{Every Pop Song Ever}
\author{Alex Grant}

\begin{document}
\maketitle
\section{Motivation}
The idea for this piece of music is for it to be ``every pop song
ever''. What do we mean by this? A typical pop song is written over a
four-chord progression, most famously I-V-vi-IV. Many other standard
four-chord progressions are in use. What if we could do all of them?

\section{Problem Statement}
We will express the idea mathematically. Let the usual seven triads
for a key be labeled as the symbols 1-7: 1 (I), 2 (ii), 3 (III), 4
(iv), 5 (V), 6 (vi), 7 (vii dim). Let $n=7$. The problem is to find
the shortest string $s\in\{1,2,\dots,n\}^N$ of length $N$ that has as
substrings every permutation of every subset of $k=4$ symbols. There
are $\binom{n}{k}k!$ such substrings, which for $n=7$ and $k=4$ is
840. 

\begin{definition}
  Let $N(n,k)$ be the length of a shortest string
  $s\in\{1,2,\dots,n\}^{\bar{N}(n,k)}$ of length $N$ that has as
  contiguous subsequences every permutation of every subset of
  $1\leq k\leq n$ symbols.
\end{definition}
In musical terms, we are looking for the shortest chord progression
that contains every possible basic four-chord progression within a
key.

Considering a (possibly impossible) sequence with maximal overlap of
the substrings,
\begin{lemma}[Lower Bound]
  \begin{equation*}
  \bar{N}(n,k) \geq \binom{n}{k}k! + k-1
\end{equation*}
\end{lemma}
For the problem at hand $\bar{N}(7,4) = 843$, which at one chord per
bar and 120 bpm would be just over seven minutes.

\section{Prelude}
Instead of substrings, consider \emph{subsequences}
(where the partiular subset does not need to appear in consecutive
indices). We'll return to substrings later.

\begin{definition}
  Let $N(n,k)$ be the length of a shortest string
  $s\in\{1,2,\dots,n\}^{N(n,k)}$ of length $N$ that has as subsequences
  every permutation of every subset of $1\leq k\leq n$ symbols.
\end{definition}

The following upper bound results from a construction due to
Savage~\cite{Sav85}. 
\begin{lemma}[Upper Bound~\cite{Sav85}]
  \begin{equation*}
    N(n,k) \leq k(n-2)+4
\end{equation*}
\end{lemma}
For $n=7$ and $k=4$, this bound evaluates to $24$ symbols. If we use
each chord for one bar, this would be a 24 bar song, which is
pleasing.

We can even be more fussy about the properties of the string: 
\begin{definition}[\cite{KouHu75}]\label{def:M}
  Let $M(n,k)$ be the length of the shortest string such that for all
  $k\leq n$, the prefix containing all permutations of any $k$-element
  subset as subsequences is as short as possible. 
\end{definition}
By finding a lower bound and a construction that meets the lower
bounds, Sch\"affer~\cite{Sch87} finds the following theorem.
\begin{theorem}[Sch\"affer~\cite{Sch87}]
  \begin{equation*}
    M(n,k) = k(n-2)+\lfloor \frac{1}{3}\left(k+1\right) \rfloor +
    3, k>2.
  \end{equation*}
\end{theorem}
Applied to the musical problem at hand, we obtain $M(7,4)=24$. So we
can even have a 24-bar song which has the property of
Definition~\ref{def:M}.  

Taking either the Savage~\cite{Sav85} or Sch\"affer~\cite{Sch87}
constructions, we can obtain a 24-bar chord progression that ``contains''
every pop sing ever. However this kind of containment is as
subsequences. The different possible progressions will be all
``interleaved''. However this might be alright if all we care about is
a short song.

\section{Fugue-I (de Bruijn style)}
Returning now to substrings rather than subsequences, the object we
may really want to consider is \emph{de Bruijn sequences}~\cite{deB46}. A
de Bruijn sequence $B(n,k)$ of order $k$ on $n$ symbols is a
\emph{cyclic} (excellent!)  sequence where every distinct length-$k$
string $t\in\{1,2,\dots,n\}^k$ appears exactly once as a
\emph{substring}. Note that this includes strings with repeated
symbols, which is not quite the problem we origianlly proposed.

A de Bruijn seqeunce $B(n,k)$ has length $n^k$, and is optimally
short (since this is also the number of distinct $k$-strings. There are 
\begin{equation*}
  \frac{\left(k!\right)^{k^{n-1}}}{k^2}
\end{equation*}
disctinct such sequences.

For $n=7, k=4$ we have $|B(7,4)|=7^4=2401$, which at one chord per beat
and 120 bpm will be just over 20 minutes.

These two concepts lend themselves to the idea of a prelude
consisting of 24 bars in $4/4$ containing every four chord progression
as a (non-contigous) subsequence, followed by the main piece.

There are a number of possible choices for meter for the main
piece. Since the length of the piece is $7^4$, a power of a prime, the
only factorisation is into multiples of $7$. However this is not very
``poppy''. 

\section{Fugue-II (Unsolved)}
To be done. Solve the actual problem stated in the introduction. As a
starting idea, form a de Bruijn graph, and delete teh vertices that
correspond to undesired substrings (ones with repetitions). The
question will then be if it possible to construct a Hamiltonian
circuit on the resulting graph.

\bibliographystyle{plain}
\bibliography{theory}

\end{document}